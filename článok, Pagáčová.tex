%Eva Pagáčová, článok
% Metódy inžinierskej práce

\documentclass[10pt,twoside,slovak,coursepaper]{article}

\usepackage[slovak]{babel}
\usepackage[IL2]{fontenc}
\usepackage[utf8]{inputenc}
\usepackage{graphicx}
\usepackage{url} % príkaz \url na formátovanie URL
\usepackage{hyperref} % odkazy v texte budú aktívne (pri niektorých triedach dokumentov spôsobuje posun textu)
\usepackage{float}
\usepackage{cite}
%\usepackage{times}

\title{Virtuálny asistent}
\author{Eva Pagáčová\\[2pt]
	{\small Slovenská technická univerzita v Bratislave}\\
	{\small Fakulta informatiky a informačných technológií}\\
	{\small \texttt{xpagacova@stuba.sk}}
	}

\date{\small 12. október 2021}

\begin{document}
\maketitle

\begin{abstract}
Virtuálny asistent sa využíva ako pomocník v domácnosti aj v pracovnom odvetví. Článok je zameraný na troch najznámejších virtuálnych asistentov založených na umelej inteligencii a to Alexa, Siri a Google Assistant. Článok sa venuje aj samotnému porovnaniu týchto virtuálnych hlasových asistentov, teda porovnaniu ich funkcií a vlastností. Obsahuje príklady využitia týchto asistentov v domácnosti aj v pracovnej oblasti.
\end{abstract}

\section{Úvod}

Virtuálni asistenti sú relatívnou novinkou vo svete, napriek tomu sa stali veľmi populárnymi v domácnostiach aj pracovných odvetviach. Funkcie virtuálnych asistentov sa vylepšujú za účelom zdokonalenia sa a praktickejšieho využitia vo svete. Výskum a vývoj v tejto technologickej oblasti neustále napreduje, najmä vďaka prirodzenej zvedavosti ľudstva a otázke, ako veľmi sme schopní posunúť hranice podobnosti s človekom. 

Na zjednodušenie určitých činností sa preto vytvoril virtuálny asistent, ktorý je schopný učiť sa sám. Využíva model strojového učenia, vďaka ktorému sa vie učiť od ľudí a zapamätať si tieto poznatky. Následne je schopný pracovať (myslieť, predvídať) samostatne, preto sa stal veľmi užitočným. Všetky stroje, ktoré ľudstvo vymyslelo by nám malo uľahčovať život a zefektívniť pracovnú či inú činnosť. Platí to aj o virtuálnych asistentoch?

\section{Virtuálny asistent}

Hoci máme v dnešnom svete všetko na dosah ruky, a jtak si chceme zľahčiť akúkoľvek prácu, ktorú máme urobiť. Sluhovia už nie sú v móde a zároveň chceme lacnejšiu alternatívu ako zamestnanca na plný úväzok. Preto sa stali virtuálni asistenti tak rýchlo populárnymi, ale ajtak ich zatiaľ nevyužívajú všetci. Niektorí ani nevedia, že existujú a netušia, aký potenciál ukrývajú. Čo je vlastne virtuálny asistent?

Virtuálny asistent môže byť človek, ale aj aplikácia. Ako človek vykonáva pomocnú prácu najmä v oblasti administratívy, zákazníckej podpory či správy firemnej stránky. Môže teda fungovať napríklad ako sekretár/ka, osobný asistent/ka alebo ako pracovník/čka zákazníckej podpory. Medzi činnosti takéhoto virtuálneho asistenta patrí napríklad: správa webovej stránky a sociálnej siete, marketingové prieskumy, komunikácia so zákazníkmi napr. cez e-mail, chat, Skype, Slack, WhatsApp, Messenger a iné. V prípade zákazníckej podpory im pomáha riešiť problémy s objednávkami a prípadné reklamácie.\cite{Bouhanikova}

Článok je však zameraný na virtuálneho asistenta ako aplikáciu, ktorá sa bežne nazýva taktiež ako AI asistent či digitálny asistent. 

\subsection{Virtuálny asistent ako inteligentný pomocník}

Virtuálny asistent je aplikácia, ktorá reaguje na hlasové povely a následne vykoná úlohy zadané používateľom. Zvyčajne sú to cloudové programy, ktoré fungujú na zariadeniach a aplikáciách s prístupom na internet. Teda sú k dispozícii na väčšine počítačov, tabletov, smartfónov a samostatných zariadení.\cite{Gupta}

Virtuálni asistenti okrem odpovedania na otázky dokážu hovoriť vtipy, prehrať hudbu, kontrolovať v dome svetlá, termostat či inteligentné domáce zariadenia (za predpokladu, že sú prepojené). Disponujú schopnosťami ľudského asistenta, teda môžu telefonovať, posielať textové správy, zapísať schôdzku, vytvoriť pripomienku, nastaviť budík, zarezervovať letenku, nájsť hotel, zapísať poznámku alebo môžu vyhľadať informácie. \cite{McLaughlin}

Vďaka tomu sa môžu naučiť zvyky, preferencie a záujmy používateľa. Stávajú sa tak užitočnejšími a efektívnejšími. Využívajú na to platformu strojového učenia, platformu spracovania prirodzeného jazyka a platformy na rozpoznávanie reči.\cite{Botelho}

\subsection{Model strojového učenia}

Strojové učenie (angl. Machine learning) je podoblasť umelej inteligencie, zaoberajúca sa metódami a algoritmami, vďaka ktorým sa program dokáže niečo naučiť. Následne vie vhodne zareagovať na dané podnety (vstupné hodnoty) bez predošlého naprogramovania odpovede na ne, reaguje len na základe naučených informácií. V daných algoritmoch sú použité prvky matematickej štatistiky, metódy štatistickej analýzy a hĺbkovú analýzu dát. \cite{Cibula}

Počítač vlastne dokáže pracovať len s presnými údajmi a faktami, ale strojové učenie mu poskytuje odhad, ktorý je pri aplikáciách strojového učenia potrebný na vytváranie predikcií.

\subsection{Využitie strojového učenia}

Strojové  učenie je veľmi užitočné napríklad pri detekcii škodlivých softvérov - malvérov (angl. malware). Algoritmy sú taktiež schopné zachytiť anomálie v prístupe k dátam na cloude a predpovedať bezpečnostné narušenie. Veľké spoločnosti ako napríklad Google využívajú strojové učenie, aby poskytli užívateľom lepšie výsledky vyhľadávania, takže užívateľ sa dostane k relevantným informáciám omnoho skôr. \cite{Labus}

\section{Najznámejší virtuálni hlasoví asistenti}

Alexa, Siri a Google Assistant sú populárni virtuálni hlasoví asistenti vytvorení spoločnosťami Amazon, Apple a Google. Je potrebné ovládať aspoň jeden svetový jazyk, aby sa používateľ vedel dorozumieť s asistentom. Nezvykne to byť problém, keďže väčšina používateľov ovláda anglický jazyk na dostatočnej úrovni. Zároveň treba klásť otázky zrozumiteľne, inak virtuálny asistent nebude reagovať korektne (aspoň používateľovi sa bude zdať, že nie).\cite{Kadlec}

\subsection{Alexa}
%doplniť
\subsection{Siri}
Siri je virtuálny hlasový asistent vytvorený pre zariadenia Apple.\cite{Strephon}
%doplniť
\subsection{Google Assistant}
%doplniť
\section{Porovnanie Alexa, Siri, Google Assistant}
%doplniť
Nakoľko sú virtuálny asistenti viac obľúbení medzi mladšou generáciou, pozrime sa na nasledujúci graf ukazujúci ich populárnosť medzi mileniálmi pri nakupovaní. Google Assistant je využívaný až 24 percentami mladých. Siri a Amazon využíva takmer rovnaký podiel mileniálov, a to 19 \% a 18 \%. \cite{Kinsella}

\begin{figure}[H]
 \centering
 \includegraphics[width=110mm]{graph}
\caption {Populárnosť virtuálnych asistentov pri nakupovaní \cite{iKinsella}}
\end{figure}
%doplniť

\section{Postrehy používateľov}

Virtuálni asistenti majú teda veľa výhod, niektoré z nich zhrnuli respondenti prieskumu zo Spojených Štátov v roku 2017. Najviac sa respondentom páčilo, že majú viac pohodlia v živote, trávia menej času pred obrazovkou (monitorom) a chválili si lepšiu orientáciu pri nakupovaní tovaru a služieb.\cite{Nae}

\begin{figure}[H]
\centering
\includegraphics[width=110mm]{graph,pluses}
\caption{Benefity virtuálneho asistenta\cite{iNae}}
\end{figure}

Títo respondenti spomenuli aj negatívne dopady virtuálnych asistentov. Polovica respondentov za riziko označila ťažkosti s kontaktovaním iných ľudí, zvýšené množstvo reklamy a nedorozumenia.\cite{Nae}

\begin{figure}[H]
\centering
\includegraphics[width=100mm]{-graf}
\caption{Riziká virtuálneho asistenta \cite{iNae}}
\end{figure}

Známy je prípad maloletého dieťaťa, ktoré si pomocou Amazon Echo kúpilo hračku bez prítomnosti rodičov. Keď sa správa objavila v televízií a moderátorka zopakovala tie isté slová ako toto dieťa, nákup spracovali aj iné zariadenia, ktoré zachytili tento príkaz z televízie.



\section{Záver}

%doplniť

% týmto sa generuje zoznam literatúry z obsahu súboru literaturakclanku.bib
\bibliography{literaturakclanku}
\bibliographystyle{abbrv}

\end{document}