% Metódy inžinierskej práce

\documentclass[10pt,twoside,slovak,coursepaper]{article}

\usepackage[slovak]{babel}
%\usepackage[T1]{fontenc}
\usepackage[IL2]{fontenc} % lepšia sadzba písmena Ľ než v T1
\usepackage[utf8]{inputenc}
\usepackage{graphicx}
\usepackage{url} % príkaz \url na formátovanie URL
\usepackage{hyperref} % odkazy v texte budú aktívne (pri niektorých triedach dokumentov spôsobuje posun textu)

\usepackage{cite}
%\usepackage{times}

\pagestyle{headings}

\title{Virtuálny asistent}
\author{Eva Pagacova\\[2pt]
	{\small Slovenská technická univerzita v Bratislave}\\
	{\small Fakulta informatiky a informačných technológií}\\
	{\small \texttt{...@stuba.sk}}
	}

\date{\small x. november 2021}

\begin{document}
\maketitle

\begin{abstract}
 Virtuálni asistenti sú relatívnou novinkou vo svete, napriek tomu sa stali veľmi populárnymi v domácnostiach aj pracovných odvetviach. Využívajú model strojového učenia, vďaka ktorému sa vedia učiť od ľudí. Tento článok je zameraný na troch najznámejších virtuálnych asistentov založených na umelej inteligencii a to Alexa, Siri a Google Assistant. Článok sa venuje aj samotnému porovnaniu týchto virtuálnych hlasových asistentov, teda porovnaniu ich funkcií a vlastností. Obsahuje príklady využitia týchto asistentov v domácnosti aj v pracovnej oblasti.
\end{abstract}

\section{Úvod} \label{uvod}
\section{Virtuálny asistent}
\subsection{Model strojového učenia}
\section{Najznámejší virtuálni hlasoví asistenti}
\subsection{Alexa}
\subsection{Siri}
\subsection{Google Assistant}
\section{Porovnanie Alexa, Siri, Google Assistant}
\section{Záver}


% týmto sa generuje zoznam literatúry z obsahu súboru literatúra.bib
\bibliography{literatura}
\bibliographystyle{alpha}
\end{document}