% Metódy inžinierskej práce

\documentclass[10pt,twoside,slovak,coursepaper]{article}

\usepackage[slovak]{babel}
%\usepackage[T1]{fontenc}
\usepackage[IL2]{fontenc} % lepšia sadzba písmena Ľ než v T1
\usepackage[utf8]{inputenc}
\usepackage{graphicx}
\usepackage{url} % príkaz \url na formátovanie URL
\usepackage{hyperref} % odkazy v texte budú aktívne (pri niektorých triedach dokumentov spôsobuje posun textu)

\usepackage{cite}
%\usepackage{times}

\pagestyle{headings}

\title{Virtuálny asistent}
\author{Eva Pagáčová\\[2pt]
	{\small Slovenská technická univerzita v Bratislave}\\
	{\small Fakulta informatiky a informačných technológií}\\
	{\small \texttt{xpagacova@stuba.sk}}
	}

\date{\small 12. october 2021}

\begin{document}
\maketitle

\begin{abstract}
 Virtuálny asistent sa využíva ako pomocník v domácnosti aj v pracovnom odvetví. Tento článok je zameraný na troch najznámejších virtuálnych asistentov založených na umelej inteligencii a to Alexa, Siri a Google Assistant. Článok sa venuje aj samotnému porovnaniu týchto virtuálnych hlasových asistentov, teda porovnaniu ich funkcií a vlastností. Obsahuje príklady využitia týchto asistentov v domácnosti aj v pracovnej oblasti.
\end{abstract}

\section{Úvod} \label{uvod}
Virtuálni asistenti sú relatívnou novinkou vo svete, napriek tomu sa stali veľmi populárnymi v domácnostiach aj pracovných odvetviach. Funkcie virtuálnych asistentov sa neustále zlepšujú za účelom zdokonalenia sa a praktickejšieho využitia vo svete. Výskum a vývoj v tejto oblasti neustále napreduje, najmä vďaka prirodzenej zvedavosti ľudstva a otázke,ako veľmi sme schopní posunúť hranice v oblasti technológie. Na zjednodušenie určitých činností sa preto vytvoril virtuálny asistent, ktorý je schopný učiť sa sám. Využíva model strojového učenia, vďaka ktorému sa vie učiť od ľudí a zapamätať si tieto poznatky. Následne je schopný pracovať (myslieť, predvídať) samostatne, preto sa stal veľmi užitočným. Všetky stroje, ktoré ľudstvo vymyslelo by nám malo uľahčovať život a zefektívniť pracovnú či inú činnosť. Platí to aj o virtuálnych asistentoch?

\section{Virtuálny asistent}
Hoci máme v dnešnom svete všetko na dosah ruky, ajtak si chceme zľahčiť akúkoľvek činnosť, ktorú máme urobiť. Sluhovia už nie sú v móde a zároveň chceme lacnejšiu alternatívu ako zamestnanca na plný úväzok. Preto sa stali virtuálni asistenti tak rýchlo populárnymi, ale ajtak ich zatiaľ nevyužívajú všetci. Niektorí ani nevedia, že existujú a netušia, aký potenciál ukrývajú. Čo je vlastne virtuálny asistent?
Virtuálny asistent môže byť človek, ale aj aplikácia. Ako človek vykonáva pomocnú prácu najmä v oblasti administratívy, zákazníckej podpory či správy firemnje stránky. Môže teda fungovať napríklad ako sekretár/ka, osobný asistent/ka alebo ako pracovník/čka zákazníckej podpory. Medzi činnosti takéhoto virtuálneho asistenta patrí napríklad: správa webovej stránky a sociálnej siete, marketingové prieskumy, komunikácia so zákazníkmi napr. cez e-mail, chat, Skype, Slack, WhatsApp, Messenger a iné. V prípade zákaznickej podpory im pomáha riešiť problémy s objednávkami a prípadné reklamácie.
V tomto článku sa však zameriam na virtuálneho asistenta ako aplikáciu, ktorá sa bežne nazýva taktiež ako AI asistent či digitálny asistent. 

\subsection{Virtuálny asistent ako inteligentný pomocník}
Virtuálny asistent je aplikácia, ktorá reaguje na hlasové povely a následne vykoná úlohy zadané používateľom. Zvyčajne sú to cloudové programy, ktoré fungujú na zariadeniach a aplikáciách s prístupom na internet. Teda sú k dispozícii na väčšine počítačov, tabletov, smartfónov a samostatných zariadení.

Virtuálni asistenti okrem odpovedania na otázky dokážu hovoriť vtipy, prehrať hudbu, kontrolovať v dome svetlá, termostat či inteligentné domáce zariadenia (za predpokladu, že sú prepojené). Disponujú schopnosťami ľudského asistenta, teda môžu telefonovať, posielať textové správy, zapísať schôdzku, vytvoriť pripomienku, nastaviť budík, zarezervovať letenku, nájsť hotel, zapísať poznámku alebo môžu vyhľadať informácie.
Vďaka tomu sa môžu naučiť zvyky, preferencie a záujmy používateľa. Stávajú sa tak užitočnejšími a využívajú na to platformu strojového učenia, spracovania prirodzeného jazyka a platforiem na rozpoznávanie reči.

\subsection{Model strojového učenia}
Strojové učenie (angl. Machine learning) je podoblasť umelej inteligencie, zaoberajúca sa metódami a algoritmami, vďaka ktorým sa program dokáže niečo naučiť. Následne vie vhodne zareagovať na dané podnety (vstupné hodnoty) bez predošlého naprogramovania odpovede na ne, reaguje len na základe naučených informácií. V daných algoritmoch sú použité prvky matematickej štatistiky, metódy štatistickej analýzy a hĺbkovú analýzu dát.
Počítač vlastne dokáže pracovať len s presnými údajmi a faktami, ale strojové učenie mu poskytuje odhad, ktorý je pri aplikáciách strojového učenia potrebný na vytváranie predikcií.

\subsubsection{Využitie strojového učenia}
Strojové  učenie je veľmi užitočné napríklad pri detekcii škodlivých softvérov - malvérov (angl. malware). Algoritmy sú taktiež schopné zachytiť anomálie v prístupe k dátam na cloude a predpovedať bezpečnostné narušenie. Veľké spoločnosti ako napríklad Google využívajú strojové učenie, aby poskytli užívateľom lepšie výsledky vyhľadávania, takže užívateľ sa dostane k relevantným informáciám omnoho skôr.

\section{Najznámejší virtuálni hlasoví asistenti}
Alexa, Siri a Google Assistant sú populárni virtuálni hlasoví asistenti vytvorení spoločnosťami Amazon, Apple a Google. 
\subsection{Alexa}
\subsection{Siri}
\subsection{Google Assistant}
\section{Porovnanie Alexa, Siri, Google Assistant}
\section{Záver}


% týmto sa generuje zoznam literatúry z obsahu súboru literatúra.bib
%\bibliography{literatura}
%\bibliographystyle{alpha}
\end{document}